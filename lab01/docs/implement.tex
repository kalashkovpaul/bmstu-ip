\chapter{Технологическая часть}

В данном разделе будут рассмотрены средства реализации, а также представлены листинги реализаций алгоритма шифрования машины <<Энигма>>.

\section{Средства реализации}
В данной работе для реализации был выбран язык программирования $C$. Данный язык удоволетворяет поставленным критериям по средствам реализации.

\section{Реализация алгоритма}

В листингах \ref{lst:enigma1} представлена реализация алгоритма шифрования машины <<Энигма>>.

\begin{center}
    \captionsetup{justification=raggedright,singlelinecheck=off}
    \begin{lstlisting}[label=lst:enigma1,caption=Реализация алгоритма шифрования машины <<Энигма>>]
char cypher(char let)
{
    if (let == ' ') let = 'X';
    if (!is_letter(let)) return let;
    if (let > 'Z') let = let - ('a' - 'A');

    int letter = (int)let;

    letter = commutation[letter - 'A'];
    letter = ((letter - 'A') + (rotors_positions[0] - 'A')) % 26;
    letter = rotors[chosen_rotors[0]][letter] - 'A';
    letter = (letter + (letter_distance(rotors_positions[1], rotors_positions[0]))) % 26;
    letter = rotors[chosen_rotors[1]][letter] - 'A';
    letter = (letter + (letter_distance(rotors_positions[2], rotors_positions[1]))) % 26;
    letter = rotors[chosen_rotors[2]][letter] - 'A';
    letter = letter - (rotors_positions[2] - 'A');
\end{lstlisting}
\end{center}

\begin{center}
    \captionsetup{justification=raggedright,singlelinecheck=off}
    \begin{lstlisting}[label=lst:enigma2,caption=Реализация алгоритма шифрования машины <<Энигма>>]
    if (letter < 0) letter = 26 + letter;

    letter = reflectors[chosen_reflector][letter] - 'A';
    letter = (letter + (rotors_positions[2] - 'A')) % 26;
    letter = find_letter_in_rotor((letter + 'A'), 2);
    letter = (letter - (letter_distance(rotors_positions[2], rotors_positions[1])));
    if (letter < 0) letter = 26 + letter;
    letter = find_letter_in_rotor((letter + 'A'), 1);
    letter = (letter - (letter_distance(rotors_positions[1], rotors_positions[0])));
    if (letter < 0) letter = 26 + letter;
    letter = find_letter_in_rotor((letter + 'A'), 0);
    letter = letter - (rotors_positions[0] - 'A');
    if (letter < 0) letter = 26 + letter;
    letter = commutation[letter];

    rotate_first_rotor();
    return (char) letter;
}
\end{lstlisting}
\end{center}


\section*{Вывод}

Были представлены листинги реализаций всех алгоритмов умножения матриц -- стандартного, Винограда и оптимизированного алгоритма Винограда. Также в данном разделе была приведена информации о выбранных средствах для разработки алгоритмов.
