\chapter{Аналитическая часть}
В этом разделе будет рассмотрен алгоритм сжатия LZW, шаги для сжатия и разжатия, а также его преимущества, недостатки и возможные способы решения возникающих проблем.


\section{Алгоритм LZW}

Алгоритм сжатия LZW (аббревиатура от фамилий \textit{Lempel}, \textit{Ziv} и \textit{Welch}) --- унверсальный алгоритм сжатия данных без потерь, в основе которого лежит использование словаря фраз.

Под словарём имеется в виду структура данных, предназначенная для объединения взаимосвязанной информации, доступ к элементам словаря предоставляется по ключу. Ключом может быть любой неизменяемый объект, число, строка, кортеж. 

При сжатии создаётся словарь фраз,в котором ключами являютсся определённые фразы (т.~е. строки или, в общем случае, последовательности бит), а значениями являются группы битоов фиксирваннй длины (например, 12-битные). Изначально словарь инициализируется всеми возможными \linebreak 1-символьными фразами (или, в общем случае, всеми фразами минимальной длины, например, всеми 8-битными фразами). 

По мере кодирования алгооритм просматривает сообщение (или последовательность бит) слева направо, и при чтении находится строка W максимальной длины, совпадающая с какой-то фразой из словаря. Она выбирается жадно, и, поскольку словарь был инициализован, она всегда найдётся. Затем код этой фразы подаётся на выход, а строка WK, где K --- следующий за W символ сообщения (или, в общем случае, последовательность бит установленной длины), вносится в словарь в качестве новой фразы, ей присваивается новый код.

Более формально алгоритм LZW можно описать следующими шагами:
\begin{enumerate}[label=\arabic*)]
	\item инициализируется словарь фраз всеми возможными последовательностями бит минимальной длин;
    \item если дошли до конца сообщения, то выдача код для W и завершение алгоритма;
	\item считывается очередную последовательность бит K установленной длины (такой, что все возможные последовательности бит данной длины содержатся в словаре);
	\item если фраза WK уже есть в словоаре, то входной фразе W присваивается значение WK, осуществляется переход к шагу 2;
	\item код для фразы W записывается в результат, фраза WK добавляется в словарь, а входной фразе W присваивается значение K, осуществляется переход к шагу 2.
\end{enumerate}

Для декодирования требуется только закодированный текст, поскольку словарь фраз может быть воссоздан непосредственно по декодируемому коду. Таким образом, результат сжатия является однозначно декодируемым: каждый раз, коогда генерируется новый код, строка добавляется в словарь фраз. Таким образом, каждая строка будет храниться в единственном экземпляре и иметь свой уникальный номер (последовательность бит).

При декодировании во время получения нового кода генерируется новая строка, и при получении известного кода строка извлекается из словаря.


\subsection{Преимущества и недостатки алгоритма LZW}

\subsubsection{Преимущества}

Простота реализации является одной из причин широкого распространения данного алгоритма. В основе алгоритма лежит использование словаря, или же префиксного дерева.

Хорошая степень сжатия, которую обеспечивает алгоритм LZW для различных типов данных, также внесла свой вклад в распространение алгоритма. Особенно высоким будет коэффициент сжатия для текстовых файлов с повторяющимися фрагментами: чем больше в сжимаемом сообщении повторяющихся фраз, тем больше места получится сэкономить при замене этих фраз на битовые последовательности из словаря.

Универсальность алгоритма LZW --- ещё одно преимущество, ведь алгоритм может быть использован для сжатия любых типов данных, включая текст, изображения, звук и видео. 

\subsubsection{Недостатки}

Эффективность метода LZW зависит от размера словаря и (косвенно) от типа данных. Больший словарь может обеспечить лучшую степень сжатия (т.к. вместит в себя больше фраз), однако требует больше памяти для хранения.

Зависимость от типа данных проявляется следующим образом: если содержимое файла содержит небольшое число повторяющихся последовательностей, эффективность сжатия может быть отрицательной: так, при сжатии уже сжатого файла, в котором повторяющиеся фрагменты отсутствуют или минимальны, в результате сжатия объём данных лишь увеличится, поскольку коротким фразам-ключам в словаре будут соответствовать превышающие их по длине значения.

Зависимость от размера словаря можно частично решить изменением максимального возможного размера словаря с числом роста фраз. Длины кодов, на которые заменяются фразы сообщения, будут расти (например, от 9 до 16 бит).  

Ещё одной проблемой, которая возникает при использовании алгоритма LZW --- ограничение по количеству фраз, которые может вместить в себя словарь. Так, при фиксированной длине в 12 бит словарь будет способен хранить всего 4096 фраз. Возможным решением данной проблемы будет введение специальной последовательности, встречая которую при разжатии, алгоритм будет очищать свой словарь и начинать заполнять его заново.

\section*{Вывод}

В данном разделе был рассмотрен алгоритм сжатия LZW, шаги для сжатия и разжатия, а также его преимущества, недостатки и возможные способы решения возникающих проблем.