\chapter{Технологическая часть}

В данном разделе рассмотрены средства реализации, реализации алгоирмтов сжатиия и разжатия LZW, а также произведено тестирование.

\section{Средства реализации}
В данной работе для реализации был выбран язык программирования $C$. Данный язык удоволетворяет поставленным критериям по средствам реализации.

\section{Реализация алгоритма}

В листингах \ref{lst:lzw-1}--\ref{lst:lzw-3} представлена реализация алгоритма сжатия LZW, а на листингах \ref{lst:decompress-1}--\ref{lst:decompress-4} --- реализация алгоритма разжатия LZW.

\begin{center}
    \captionsetup{justification=raggedright,singlelinecheck=off}
    \begin{lstlisting}[label=lst:lzw-1,caption=Реализация алгоритма сжатия LZW (часть 1)]
ssize_t lzw_compress(struct lzw_state *state, uint8_t *src, size_t slen, uint8_t *dest, size_t dlen) 
{
	if (state->was_init == false) 
	{
		lzw_init(state);
		lzw_output_code(state, CODE_CLEAR);
	}
	code_t code = CODE_EOF;
	size_t prefix_end = 0;
	state->wptr = 0;

	while (state->rptr + prefix_end < slen) 
	{
		if (state->wptr + (state->tree.code_width >> 3) + 1 + 2 + 2 > dlen) 
			return state->wptr;
\end{lstlisting}
\end{center}


\begin{center}
    \captionsetup{justification=raggedright,singlelinecheck=off}
    \begin{lstlisting}[label=lst:lzw-2,caption=Реализация алгоритма сжатия LZW (часть 2)]
	
		++prefix_end;
		bool overlong = ((state->longest_prefix_allowed > 0) && (prefix_end >= state->longest_prefix_allowed));
		bool existing_code = lzw_string_table_lookup(state, src + state->rptr, prefix_end, &code);
		if (!existing_code || overlong)
        {
			uint8_t symbol = src[state->rptr + prefix_end - 1];
			code_t parent = code;
			code_t parent_len = 1 + lzw_node_prefix_len(state->tree.node[parent]);
			lzw_output_code(state, parent);

			if (state->tree.next_code == (1UL << state->tree.code_width)
#if LZW_MAX_CODE_WIDTH == 16
				|| (state->tree.next_code == LZW_MAX_CODES - 1)
#endif
            )
            {
				if (state->tree.code_width < LZW_MAX_CODE_WIDTH)
                {
					++state->tree.code_width;
				}
                else
                {
					lzw_flush_reservoir(state, dest, false);
					lzw_output_code(state, CODE_CLEAR);
					lzw_reset(state);
					lzw_flush_reservoir(state, dest, false);
					state->tree.next_code = CODE_EOF;
				}
			}
			state->tree.node[state->tree.next_code++] = lzw_make_node(symbol, parent, parent_len);

			if (parent_len > state->longest_prefix)
				state->longest_prefix = parent_len;
\end{lstlisting}
\end{center}


\begin{center}
    \captionsetup{justification=raggedright,singlelinecheck=off}
    \begin{lstlisting}[label=lst:lzw-3,caption=Реализация алгоритма сжатия LZW (часть 3)]
			state->rptr += parent_len;
			prefix_end = 0;
			lzw_flush_reservoir(state, dest, false);
		}
	}
	if (prefix_end != 0)
    {
		lzw_output_code(state, code);
		lzw_flush_reservoir(state, dest, false);
		state->rptr += prefix_end;
		prefix_end = 0;
	}

	if ((state->rptr + prefix_end == slen && state->tree.prev_code != CODE_EOF)
		|| (state->wptr == 0 && state->bitres_len > 0))
    {
		lzw_output_code(state, CODE_EOF);
		lzw_flush_reservoir(state, dest, true);
	}
	return state->wptr;
}
\end{lstlisting}
\end{center}

\begin{center}
    \captionsetup{justification=raggedright,singlelinecheck=off}
    \begin{lstlisting}[label=lst:decompress-1,caption=Реализация алгоритма разжатия  LZW (часть 1)]
ssize_t lzw_decompress(struct lzw_state *state, uint8_t *src, size_t slen, uint8_t *dest, size_t dlen)
{
	if (state->was_init == false)
		lzw_init(state);
	uint32_t bitres = state->bitres;
	uint32_t bitres_len = state->bitres_len;

	uint32_t code = 0;
	size_t wptr = 0;

	while (state->rptr < slen)
    {
\end{lstlisting}
\end{center}


\begin{center}
    \captionsetup{justification=raggedright,singlelinecheck=off}
    \begin{lstlisting}[label=lst:decompress-2,caption=Реализация алгоритма разжатия  LZW (часть 2)]
		while ((bitres_len < state->tree.code_width) && (state->rptr < slen))
        {
			bitres |= src[state->rptr++] << bitres_len;
			bitres_len += 8;
		}

		state->bitres = bitres;
		state->bitres_len = bitres_len;

		if (state->bitres_len < state->tree.code_width)
        {
			return LZW_INVALID_CODE_STREAM;
		}

		code = bitres & mask_from_width(state->tree.code_width);
		bitres >>= state->tree.code_width;
		bitres_len -= state->tree.code_width;

		if (code == CODE_CLEAR)
        {
			if (state->tree.next_code != CODE_FIRST)
				lzw_reset(state);
			continue;
		} else if (code == CODE_EOF)
        {
			break;
		} else if (state->must_reset)
        {
			return LZW_STRING_TABLE_FULL;
		}

		if (code <= state->tree.next_code)
        {
			bool known_code = code < state->tree.next_code;
			code_t tcode = known_code ? code : state->tree.prev_code;
			size_t prefix_len = 1 + lzw_node_prefix_len(state->tree.node[tcode]);
\end{lstlisting}
\end{center}

\begin{center}
    \captionsetup{justification=raggedright,singlelinecheck=off}
    \begin{lstlisting}[label=lst:decompress-3,caption=Реализация алгоритма разжатия  LZW (часть 3)]
			uint8_t symbol = 0;
			if (!known_code && state->tree.prev_code == CODE_EOF)
				return LZW_INVALID_CODE_STREAM;

			if (prefix_len > state->longest_prefix)
				state->longest_prefix = prefix_len;

			if (prefix_len + (known_code ? 0 : 1) > dlen)
				return LZW_DESTINATION_TOO_SMALL;

			if (wptr + prefix_len + (known_code ? 0 : 1) > dlen)
				return wptr;

			for (size_t i=0 ; i < prefix_len ; ++i)
            {
				symbol = lzw_node_symbol(state->tree.node[tcode]);
				dest[wptr + prefix_len - 1 - i] = symbol;
				tcode = lzw_node_parent(state->tree.node[tcode]);
			}
			wptr += prefix_len;

			if (state->tree.prev_code != CODE_EOF)
            {
				if (!known_code) {
					dest[wptr++] = symbol;
				}

				state->tree.node[state->tree.next_code] = lzw_make_node(symbol, state->tree.prev_code, 1 + lzw_node_prefix_len(
					state->tree.node[state->tree.prev_code]
				));

				if (state->tree.next_code >= mask_from_width(state->tree.code_width)) {
					if (state->tree.code_width == LZW_MAX_CODE_WIDTH)
                    {
						state->must_reset = true;
						state->tree.prev_code = code;
\end{lstlisting}
\end{center}

\begin{center}
    \captionsetup{justification=raggedright,singlelinecheck=off}
    \begin{lstlisting}[label=lst:decompress-4,caption=Реализация алгоритма разжатия  LZW (часть 4)]
						continue;
					}
					++state->tree.code_width;
				}
				state->tree.next_code++;
			}
			state->tree.prev_code = code;
		}
        else
			return LZW_INVALID_CODE_STREAM;
	}
	return wptr;
}
\end{lstlisting}
\end{center}

\section{Тестирование}

Тестирование разработанной программы производилось следующим образом: для файлов различных форматов производилось сжатие и разжатие, проверка совпадения исходного значения файла и получившегося, а также расчёт коэффициента сжатия для конкретного файла.


\begin{table}[h]
	\begin{center}
		\begin{threeparttable}
		\captionsetup{justification=raggedright,singlelinecheck=off}
		\caption{\label{tbl:functional_test} Функциональные тесты}
		\begin{tabular}{|c@{\hspace{7mm}}|c@{\hspace{7mm}}|c@{\hspace{7mm}}|c@{\hspace{7mm}}|c@{\hspace{7mm}}|c@{\hspace{7mm}}|}
			\hline
			Формат файла & Размер файла, байты & Коэффициента сжатия, \% \\ 
			\hline
			
			TXT & 376844 & 75.75 \\ \hline
			PNG & 16760 & -33.28 \\ \hline
			BMP & 818058 & 4.90\\ \hline
			MP3 & 733645 & -31.21 \\ \hline
			JS &  470167 & 68.65 \\ \hline
		\end{tabular}
		\end{threeparttable}
	\end{center}
	
\end{table}


\section*{Вывод}

В данном разделе были  рассмотрены средства реализации, реализации алгоирмтов сжатиия и разжатия LZW, а также произведено тестирование.

