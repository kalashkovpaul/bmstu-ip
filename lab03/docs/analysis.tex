\chapter{Аналитическая часть}
В этом разделе будут рассмотрен шифровальный алгоритм AES, а также его работа в режиме PCBC.


\section{Алгоритм AES}

Шифровальная алгоритм AES (англ. \textit{Advanced Encryption Standart} --- AES) --- симметричный блочнный шифровальный алгоритм, разработанный в 2001 году Национальный институтом стандартов и технологий США. Он использует блочное шифрование, длина блока фиксирована и равна 128 битам, длина ключа 128, 192 либо же 256 бит. Он состоит раундов шифрования,количество которых зависит от длины ключа: 10 раундов для ключа размером 128 бит, 12 раундов для ключа размером 192 бита и 14 раундов для ключа размером 256 бит.

Прежде чем перейти к раундам шифрования, происходит генерация ключей раунда (раундовых ключей) из исходного ключа, Рассмотрим, как это происходит.

\subsection{Получение ключей раунда}

Определим фунцию $g$, изменяющую четырёхбайтовое слово так, как указано на рисунке \ref{img:g_function}.

\imgScale{0.5}{g_function}{Схема функции g}


Ключей раундов $k_{i}$ необходимо на 1 больше, чем количество раундов, т.е. 11 ключей раундов для основногоключа длиной 128 бит, 13 ключей раунда для основного ключа длиной 192 бита и 15 ключей раунда для основного ключа длинной 256 бит.

Алгоритм получения ключа раунда из исходного ключа преставлен в виде схемы алгоритма на рисунке \ref{img:round_keys}.

\imgScale{0.5}{round_keys}{Схема функции g}

\clearpage
\subsection{Раунд шифрования}

Раунд шифрования состоит из 5 следующих этапов
\begin{enumerate}[label=\arabic*)]
	\item замена (англ. \textit{confussion});
    \item процедура перестановки строк (англ. \textit{row-row mix procedure} --- RR);
	\item процедура перестановки столбцов (англ. \textit{row-columns mix} --- RC);
	\item смешивание ключа (англ. \textit{key mixing} --- KM).
\end{enumerate}

Замена обеспечивает нелинейность алгоритма шифрования, обрабатываая каждый байт состояния, производя нелинейную замену байт с использованием таблицы замен.

Процедура перестановки строк представляет из себя циклический сдвиг строки ссостояний на количество байт, зависящее от номера строки.

Процедура перестановки столбцов 4 байта каждого столбца смешиваются с использовоанием обратимой линейной трансформации. На последнем раунду эта процедура не выполняется.

Смешивание ключа представляет из себя операцию XOR с ключом раунда, полученным заранее.


\section{Режимы работы алгоритма AES}

Режим шифрования --- метод применения блочного шифра, позволяющий преобразовать последовательность блоков открытых данных в последовательность блоков зашифрованных данных.

Для AES рекомендованы следующие режими работы:
\begin{enumerate}[label=\arabic*)]
	\item режим электронной кодовой книги (англ. \textit{Electronic Code Bloc} --- ECB);
    \item режим сцепления блоков (англ. \textit{Cipher Block Chaining} -- CBC);
	\item режим параллельного сцепления блоков (англ. \textit{Parallel Cipher Block Chaining} -- PCBC);
	\item режим обратной связи по шифротексту (англ. \textit{Cipher Feed Back} --- CFB);
	\item режим обратной связи по выходу (англ. \textit{Output Feed Back} --- OFB).
\end{enumerate}

В данной работе будет рассмотрен режим обратной связи по шифротексту (CFB).

\subsection{Режим параллельного сцепления блоков}

В данном режиме используется вектор исполнения (англ. \textit{Initialization vector} --- IV)--- случайная последовательность символов, которую добавляют к ключу шифрования для повышения его безопасности. Он затрудняет определение закономерностей в рядах данных и делает их более устойчивыми ко взлому.

В режиме PCBC вектор исполнения IV подвергается операции XOR с фрагментом открытого текста, результат операции шифруется при помощи алгоритма AES. Полученное значение является фрагментом шифротекста. После этого оно подвергается операции XOR с фрагментом открытого текста, рузельтат операции становится новым значением вектора IV.

Если происходит расшифровка, фрагмент шифротекста расшифровывается при помощи алгоритма AES,после чего подвергается операции XOR с вектором IV. Полученное значение является фрагментом открытого текста. Оно подвергается операции XOR с фрагментом шифротекста,  результат операции становится новым значением вектора IV.

\section*{Вывод}

В данном разделе был рассмотрен шифровальный алгоритм AES, его составляющие и режимы работы, а также режим параллельного сцепления блоков (PCBC).

