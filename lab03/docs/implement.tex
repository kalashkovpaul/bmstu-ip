\chapter{Технологическая часть}

В данном разделе будут рассмотрены средства реализации, а также представлены листинги реализации шифровального алгоритма AES и режима работы PCBC, а также произведено тестирование.

\section{Средства реализации}
В данной работе для реализации был выбран язык программирования $C$. Данный язык удоволетворяет поставленным критериям по средствам реализации.

\section{Реализация алгоритма}

В листингах \ref{lst:aes1}--\ref{lst:aes2} представлена реализация шифровального алгоритма AES, на листинге \ref{lst:pcbc} --- реализация режима работы PCBC.

\begin{center}
    \captionsetup{justification=raggedright,singlelinecheck=off}
    \begin{lstlisting}[label=lst:aes1,caption=Реализация шифровального  алгоритма AES, шифровка]
void EncryptAES128(const byte *msg, const byte *key, byte *c) {
    int i;
    byte keys[176];
    expand_key128(key,keys);
    memcpy(c, msg, 16);
    xor_round_key(c,keys,0);
    for(i=0; i<9; i++) {
        sub_bytes(c,16);
        shift_rows(c);
        mix_cols(c);
        xor_round_key(c, keys, i+1);
    }
    sub_bytes(c,16);
    shift_rows(c);
    xor_round_key(c, keys, 10);
}
\end{lstlisting}
\end{center}


\begin{center}
    \captionsetup{justification=raggedright,singlelinecheck=off}
    \begin{lstlisting}[label=lst:aes2,caption=Реализация шифровального  алгоритма AES расшифровка]
void DecryptAES128(const byte *c, const byte *key, byte *m) {
    int i;

    byte keys[176];
    expand_key128(key,keys);

    memcpy(m,c,16);
    xor_round_key(m,keys,10);
    shift_rows_inv(m);
    sub_bytes_inv(m, 16);

    for (i=0; i<9; i++) {
        xor_round_key(m,keys,9-i);
        mix_cols_inv(m);
        shift_rows_inv(m);
        sub_bytes_inv(m, 16);
    }
    xor_round_key(m, keys, 0);
}
\end{lstlisting}
\end{center}

\clearpage

\begin{center}
    \captionsetup{justification=raggedright,singlelinecheck=off}
    \begin{lstlisting}[label=lst:pcbc,caption=Реализация режима работы PCBC]
void pcbc(byte input128[], byte output128[], byte mode) {
    if (mode == 'e')
    {
        byte to_cypher[16] = {0};
        for (int i = 0; i < 16; i++)
            to_cypher[i] = IV[i] ^ input128[i];

        EncryptAES128(to_cypher, key, output128);
        for (int i = 0; i < 16; i++)
            IV[i] = input128[i] ^ output128[i];
    }
    else
    {
        byte almost_decyphered[16] = {0};
        DecryptAES128(input128, key, almost_decyphered);
        for (int i = 0; i < 16; i++)
        {
            output128[i] = IV[i] ^ almost_decyphered[i];
            IV[i] = input128[i] ^ output128[i];
        }
    }
}
\end{lstlisting}
\end{center}

\section{Тестирование}

Тестирование разработанной программы производилось следующим образом: выбирались случайные значения ключа и вектора IV, а также получалась случайная последовательность блоков для шифрования длиной $n$. Она зашифровывалась и расшифровывалась, проверялось совпадение полученного результата с начальными данными. Данная процедура повторялась $n$ раз для значений $n$ от 1 до 100.

\begin{table}[h]
	\begin{center}
		\begin{threeparttable}
		\captionsetup{justification=raggedright,singlelinecheck=off}
		\caption{\label{tbl:functional_test} Функциональные тесты}
		\begin{tabular}{|c@{\hspace{7mm}}|c@{\hspace{7mm}}|c@{\hspace{7mm}}|c@{\hspace{7mm}}|c@{\hspace{7mm}}|c@{\hspace{7mm}}|}
			\hline
			Длина, байты & Шифруемое значение & Результат работы \\ 
			\hline
			
			8 &
		    12345678 &
			Сообщение об ошибке \\ \hline
			16 & 1234567812345678 &
			\begin{tabular}{c}
			cad29cf5b295a4bf\\
			5905026c48d83c5
			\end{tabular}   \\ \hline

			32 &
			\begin{tabular}{c}
			1234567812345678\\
			1234567812345678
			\end{tabular} &
			\begin{tabular}{c}
			cad29cf5b295a4bf\\
			590d5026c48d83c5\\
			9ab00f0ae0135012\\
			710c4ba8595b138c
			\end{tabular} \\ \hline

		\end{tabular}
		\end{threeparttable}
	\end{center}
	
\end{table}


\section*{Вывод}

В данном разделе были рассмотрены средства реализации, а также представлены листинги реализации шифровального алгоритма AES и режима работы PCBC,  произведено тестирование.

