\chapter{Аналитическая часть}
В этом разделе будут рассмотрен шифровальный алгоритм DES, а также его работа в режиме обратной связи  по шифротексту (CFB).


\section{Алгоритм DES}

Шифровальная алгоритм DES (англ. \textit{Data Encryption Standart} --- DES) --- симметричный шифровальный алгоритм, разработанный в 1977 году компанией IBM. Он использует блочное шифрование, длина блока фиксирована и равна 64 битам. Он состоит из 3 следующих шагов: начальная перестановка (англ. \textit{Initial Permutation} --- IP), во время которой биты переставляются в порядке, определённом в специальной таблице, 16 раундов шифрования, а также завершающей перестановки (англ. \textit{Final Permutation} -- FP), соовершающей преобразования, обратные сделанным на первом шаге. Рассмотрим подробнее раунд шифрования.

\subsection{Раунд шифрования}

Раунд шифрования состоит из 5 следующих этапов
\begin{enumerate}[label=\arabic*)]
	\item расширение (англ. \textit{expansion} --- E);
    \item получение ключа раунда (англ. \textit{Round Key} -- RK);
	\item скремблирование (англ. \textit{substitution} --- S);
	\item перестановка (англ. \textit{permutation} --- P)
	\item смешивание ключа (англ. \textit{key mixing} --- KM).
\end{enumerate}

Расширение, во время которого каждая из половин блока шифрования по 32 бит дополняется путём перестановки и дублировоания бит до длины в 48 бит.

Получение ключа раунда необходимо для применения в раунде шифрования 48-битного ключа раунда, полученного из основного ключа DES. Основной ключ имеет длину 64 бита, однако значащих бит из 64 всего 56, остальные добавлены для избыточности и контроля передачи ключа. Из этих 56 бит получают 48 путём разбиения на равные части и применению битовой операции циклического сдвига и нахождению нового значения посредством специальной таблицы.

Скремблирование предназначено для получения из 48-битного потока 32-битного путём разбиения на 6 частей по 8 бит и обработки каждой части в S-блоках (англ. \textit{Substitution boxes}), которые заменяют блоки с длиной 6 бит на блоки 4 бит посредством использования специальной таблицы.

Перестановка представляет из себя перемешивания полученной последовательности из 32 бит при помощи таблицы перемешивания.

Смешивание ключа представляет из себя операцию XOR полученного 32-битного значения c ключом раунда.


\section{Режимы работы алгоритма DES}

Режим шифрования --- метод применения блочного шифра, позволяющий преобразовать последовательность блоков открытых данных в последовательность блоков зашифрованных данных.

Для DES рекомендованы следующие режими работы:
\begin{enumerate}[label=\arabic*)]
	\item режим электронной кодовой книги (англ. \textit{Electronic Code Bloc} --- ECB);
    \item режим сцепления блоков (англ. \textit{Cipher Block Chaining} -- CBC);
	\item режим параллельноого сцепления блоков (англ. \textit{Parallel Cipher Block Chaining} -- PCBC);
	\item режим обратной связи по шифротексту (англ. \textit{Cipher Feed Back} --- CFB);
	\item режим обратной связи по выходу (англ. \textit{Output Feed Back} --- OFB).
\end{enumerate}

В данной работе будет рассмотрен режим обратной связи по шифротексту (CFB).

\subsection{Режим обратной связи по шифротексту}

В данном режиме используется вектор исполнения (англ. \textit{Initialization vector} --- IV)--- случайная последовательность символов, которую добавляют к ключу шифрования для повышения его безопасности. Он затрудняет определение закономерностей в рядах данных и делает их более устойчивыми ко взлому.

В режиме CFB вектор исполнения IV зашифровывается при помощи алгоритма DES с использованием основного ключа. После этого, если происходит зашифровка, полученное значение подвергается операции XOR с блоком, который нужно зашифровать. Полученное значение является зашифрованным блоком данных и значением IV для следующего шифрования --- это обеспечивает взаимосвязь блоков данных.

Если происходит расшифровка, полученное значение подвергается операции XOR с блоком, который нужно расшифровать. Полученное значение является расшифрованным блоком данных, а блок зашифрованного текста явлляется следующим значением IV.

\section*{Вывод}

В данном разделе был рассмотрен шифровальный алгоритм DES, его составляющие и режимы работы, а также режим обратной связи по шифротексту (CFB).

