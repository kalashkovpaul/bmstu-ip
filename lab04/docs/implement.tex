\chapter{Технологическая часть}

В данном разделе будут рассмотрены средства реализации, а также представлены листинги реализации криптографического алгоритма RSA и алгоритма хеширования MD5, а также произведено тестирование.

\section{Средства реализации}
В данной работе для реализации был выбран язык программирования $C$. Данный язык удоволетворяет поставленным критериям по средствам реализации.

\section{Реализация алгоритма}

В листингах \ref{lst:rsa1}--\ref{lst:rsa2} представлена реализация криптографического алгоритма RSA, на листингах \ref{lst:md5-1}--\ref{lst:md5-2} --- реализация алгоритма хеширования MD5.

\begin{center}
    \captionsetup{justification=raggedright,singlelinecheck=off}
    \begin{lstlisting}[label=lst:rsa1,caption=Реализация алгоритма получения ключей RSA часть 1]
void generate_keys(private_key* ku, public_key* kp)
{
    char buf[BUFFER_SIZE];
    int i;
    mpz_t phi; mpz_init(phi);
    mpz_t tmp1; mpz_init(tmp1);
    mpz_t tmp2; mpz_init(tmp2);

    mpz_set_ui(ku->e, 3);

    for(i = 0; i < BUFFER_SIZE; i++)
        buf[i] = rand() % 0xFF;
    buf[0] |= 0xC0;
    buf[BUFFER_SIZE - 1] |= 0x01;
    mpz_import(tmp1, BUFFER_SIZE, 1, sizeof(buf[0]), 0, 0, buf);
    mpz_nextprime(ku->p, tmp1);
    mpz_mod(tmp2, ku->p, ku->e);
\end{lstlisting}
\end{center}


\begin{center}
    \captionsetup{justification=raggedright,singlelinecheck=off}
    \begin{lstlisting}[label=lst:rsa2,caption=Реализация алгоритма получения ключей RSA часть 2]
  while(!mpz_cmp_ui(tmp2, 1))
    {
        mpz_nextprime(ku->p, ku->p);   
        mpz_mod(tmp2, ku->p, ku->e);
    }

    do {
        for(i = 0; i < BUFFER_SIZE; i++)
            buf[i] = rand() % 0xFF;
        buf[0] |= 0xC0;
        buf[BUFFER_SIZE - 1] |= 0x01;
        mpz_import(tmp1, (BUFFER_SIZE), 1, sizeof(buf[0]), 0, 0, buf);
        mpz_nextprime(ku->q, tmp1);
        mpz_mod(tmp2, ku->q, ku->e);
        while(!mpz_cmp_ui(tmp2, 1))
        {
            mpz_nextprime(ku->q, ku->q);
            mpz_mod(tmp2, ku->q, ku->e);
        }
    } while(mpz_cmp(ku->p, ku->q) == 0);

    mpz_mul(ku->n, ku->p, ku->q);

    mpz_sub_ui(tmp1, ku->p, 1);
    mpz_sub_ui(tmp2, ku->q, 1);
    mpz_mul(phi, tmp1, tmp2);

    if(mpz_invert(ku->d, ku->e, phi) == 0)
    {
        mpz_gcd(tmp1, ku->e, phi);
        printf("gcd(e, phi) = [%s]\n", mpz_get_str(NULL, 16, tmp1));
        printf("Invert failed\n");
    }

    mpz_set(kp->e, ku->e);
    mpz_set(kp->n, ku->n);
\end{lstlisting}
\end{center}

\clearpage

\begin{center}
    \captionsetup{justification=raggedright,singlelinecheck=off}
    \begin{lstlisting}[label=lst:md5-1,caption=Реализация алгоритма хеширования MD5 часть 1]
void md5(const uint8_t *initial_msg, size_t initial_len, uint8_t *digest) {
    uint32_t h0, h1, h2, h3;
    uint8_t *msg = NULL;
    size_t new_len, offset;
    uint32_t w[16];
    uint32_t a, b, c, d, i, f, g, temp;
    h0 = 0x67452301;
    h1 = 0xefcdab89;
    h2 = 0x98badcfe;
    h3 = 0x10325476;
    for (new_len = initial_len + 1; new_len % (512/8) != 448/8; new_len++);
    msg = (uint8_t*)malloc(new_len + 8);
    memcpy(msg, initial_msg, initial_len);
    msg[initial_len] = 0x80; // append the "1" bit; most significant bit is "first"
    for (offset = initial_len + 1; offset < new_len; offset++)
        msg[offset] = 0; // append "0" bits
    to_bytes(initial_len*8, msg + new_len);
    to_bytes(initial_len>>29, msg + new_len + 4);
    for(offset=0; offset<new_len; offset += (512/8)) {
        for (i = 0; i < 16; i++)
            w[i] = to_int32(msg + offset + i*4);
        a = h0;
        b = h1;
        c = h2;
        d = h3;
        for(i = 0; i<64; i++) {
            if (i < 16) {
                f = (b & c) | ((~b) & d);
                g = i;
            } else if (i < 32) {
                f = (d & b) | ((~d) & c);
                g = (5*i + 1) % 16;
            } else if (i < 48) {
                f = b ^ c ^ d;
                g = (3*i + 5) % 16;
\end{lstlisting}
\end{center}

\clearpage

\begin{center}
    \captionsetup{justification=raggedright,singlelinecheck=off}
    \begin{lstlisting}[label=lst:md5-2,caption=Реализация алгоритма хеширования MD5 часть 2]
            } else {
                f = c ^ (b | (~d));
                g = (7*i) % 16;
            }
            temp = d;
            d = c;
            c = b;
            b = b + LEFTROTATE((a + f + k[i] + w[g]), r[i]);
            a = temp;
        }
        h0 += a;
        h1 += b;
        h2 += c;
        h3 += d;
    }
    free(msg);
    to_bytes(h0, digest);
    to_bytes(h1, digest + 4);
    to_bytes(h2, digest + 8);
    to_bytes(h3, digest + 12);
}
\end{lstlisting}
\end{center}


\section{Тестирование}

Тестирование разработанной программы производилось следующим образом: выбирались случайные значения содержимого файла длиной $n$. Для данного содержимого файла составлялсь подпись, после чего осуществлялась её проверка. Данная процедура повторялась $n$ раз для значений $n$ от 1 до 100.

\clearpage

\begin{table}[h]
	\begin{center}
		\begin{threeparttable}
		\captionsetup{justification=raggedright,singlelinecheck=off}
		\caption{\label{tbl:functional_test} Функциональные тесты}
		\begin{tabular}{|c@{\hspace{7mm}}|c@{\hspace{7mm}}|c@{\hspace{7mm}}|c@{\hspace{7mm}}|c@{\hspace{7mm}}|c@{\hspace{7mm}}|}
			\hline
			Длина, байты & Шифруемое значение & Результат работы \\ 
			\hline
			
			8 &
		    12345678 &
			Сообщение об ошибке \\ \hline
			16 & 1234567812345678 &
			\begin{tabular}{c}
			cad29cf5b295a4bf\\
			5905026c48d83c5
			\end{tabular}   \\ \hline

			32 &
			\begin{tabular}{c}
			1234567812345678\\
			1234567812345678
			\end{tabular} &
			\begin{tabular}{c}
			cad29cf5b295a4bf\\
			590d5026c48d83c5\\
			9ab00f0ae0135012\\
			710c4ba8595b138c
			\end{tabular} \\ \hline

		\end{tabular}
		\end{threeparttable}
	\end{center}
	
\end{table}


\section*{Вывод}

В данном разделе были рассмотрены описания модулей программы, а также представлены схемы алгоритма RSA, и алгоритма хеширования MD5 и проведено функциональное тестирование.

