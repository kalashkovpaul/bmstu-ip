\chapter{Аналитическая часть}
В этом разделе будут рассмотрен криптографический алгоритм RSA, алгоритм хеширования MD5, понятие электронной подписи и принципы её получения и проверки с использованием алгоритмов RSA и MD5.


\section{Алгоритм RSA}

Криптографический алгоритм RSA (аббревиатура от фамилий \textit{Rivest}, \linebreak\textit{Shamir} и \textit{Adleman}) --- ассиметричный криптографический алгоритм, в основе которогоо лежит сложность задачи факторизациии произведения двух взаимно простых чисел.

Для шифрования используется операция возведения в степень по модулю большого числа, а для дешифрации --- вычисление функции Эйлера от этого большого числа за разумное время, что можно осуществить при наличии информации о разложении данного большого числа на простые множители.

В криптографической системе с открытым ключом каждый участник располагает как открытым ключом (англ. \textit{public key}), так и закрытым ключом (англ. \textit{private key}). В криптографической системе RSA каждый ключ состоит из пары целых чисел. 


RSA ключи генерируются следующим образом:
\begin{enumerate}[label=\arabic*)]
	\item выбираются два отличающихся друг от друга случайных простых числа $p$ и $q$, лежащие в установленнном диапазоне;
    \item вычисляется их произведение $n = p \cdot q$, называемое модулем;
	\item вычисляется значение функции Эйлера от числа $n$: $\phi(n) = (p - 1)\cdot (q - 1)$;
	\item выбирается целое число $e$ ($1 < e < \phi(n)$), взаимно простое со значением $\phi(n)$, оно называется открытой экспонентой;
	\item вычисляется число $d  = e^{-1} mod (\phi(n))$, оно называется закрытой экспонентой.
\end{enumerate}

Пара $(e, n)$ публикуются в качестве открытого ключа RSA, а пара $(d, n)$ --- в виде закрытого ключа.


Шифрование сообщения $m$ ($0 < m < n - 1)$ в зашифрованное сообщение $c$ производится по формуле $ c = E(m, k_1) = E(m, n, e) = m^{e} mod (n)$.

Дешифрация: $m = D(c, k_2) = D(c, n, d) = c^{d} mod (n)$

У данного принципа имеется следующие минусы:
\begin{enumerate}[label=\arabic*)]
	\item если $m = 0$, то и $c = 0$;
	\item если $m_1 = m_2$, то и $c_1 = c_2$.
\end{enumerate}

Из-за этого RSA используется для передачи ключей других шифров.


\section{Алгоритм MD5}

MD5 (англ. \textit{Message Digest 5}) --- алгоритм хеширования, предназначенный для получения последовательнсти длиной 128 бит, используемой для последующей проверки подлинности сообщений произвольных длины.

На вход алгоритма поступает последовательность бит произвольной длины $L$, хеш которй нужно найти. 
Алгоритм MD5 состоит из 4 следующих этапов
\begin{enumerate}[label=\arabic*)]
	\item выравнивание потоков;
    \item добавление длины сообщения;
	\item инициализация буфера;
	\item вычисления в цикле.
\end{enumerate}

Выравнивание потоков представляет из себя добавление некоторого числа нулевых бит такое, чтобы новая длина последовательности $L'$ стала сравнима с 448 по модулю 512. Выравнивание происходит в любом случае, даже если длина исходного потока уже сравнима с 448

Под добавлением длины сообщения представляет из себя добавление 64 битов в последовательность: сначала младшие 4 байта, потом старшие 4 байта. После этого длина потока станет кратной 512. Вычисления будут основываться на представлении этого потока данных в виде массива слов по 512 бит.

После этого происходит инициализация буфера, состоящего из 4-х переменных размерностью 32 бита, начальные значения которых задаются шестнадцатеричными числами.
В этих переменных будут храниться результаты промежуточных вычислений.

Далее в цикле каждый блок длиной 512 бит проходит 4 этапа вычислений по 16 раундов. Для этого блок представляется в виде массива $X$ из 16 слов по 32 бита.
Все раунды однотипны и имеют вид: [abcd k s i], определяемый как $a = b + ((a +Fun(b, c,d) + X[k] + T[i]	) << s)$, где $k$ --- номер 32-битного слова из текущего блока, число $s$ задаётся отдельно для каждого раунда, $T$ --- таблица констант.

Результат вычислений хранится в переменных  $a$, $b$, $c$ и $d$.

\section{Электронная подпись}

Электронная (цифровая) подпись --- ЭП --- позволяет подвердить авторство электронного документа. Она связана не только с автором документа, но и с самим документов (при помощи криптографических методов) и не может быть подделана при поммощи обычного копирования.

Создание ЭП с использованием криптографического алгоритма RSA и алгоритма хеширования MD5 происходит следующим образом:
\begin{enumerate}[label=\arabic*)]
	\item происходит хеширование сообщения при помощи MD5, сообщение --- файл, который неообходимо подписать;
    \item происходит шифрование с использованием закрытого ключа RSA последовательности 128 бит, полученных на предыдущем этапе;
	\item значение подписи --- результат шифрования.
\end{enumerate}

Проверка ЭП с использованием криптографического алгоритма RSA и алгоритма хеширования MD5 происходит следующим образом:
\begin{enumerate}[label=\arabic*)]
	\item происходит хеширование сообщения при помощи MD5, сообщение --- файл, подпись которого необходимо проверить;
    \item происходит дешифрация подписи с использованием открытого ключа RSA;
	\item происходит побитовая сверка значений, полученных на предыдущих этапах, если они одинаковы, подпись считается подлинной.
\end{enumerate}

\section*{Вывод}

В данном разделе был рассмотрен криптографический алгоритм RSA, алгоритм хеширования MD5, а также понятие электронной подписи и принципы её получения и проверки с использованием алгоритмов RSA и MD5.

